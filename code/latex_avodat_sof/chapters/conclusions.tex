
הצעת המחקר הזו מתבססת על תיאור של אדם טלר, אחד ההיסטריונים הכלכליים של יהודי פולין המשפיעים ביותר, בספר "כח, כסף והשפעה", טלר הציג ביקורת על יישום של תאוריה כלכלית על נתונים היסטוריים, כאשר הטענה המרכזית שלו היא שהתמקדות בתיאור סיבתי מפספסת את השאלות אודות הקונטקסט. 

אם הצעת המחקר הזו אכן ניתנת לביצוע, הרי שניתן להראות שאפשר לענות על שאלות היסטוריות רבות בצורה מעמיקה, באמצעות התמקדות בתאוריה, ובאמצעות בידוד גורמים סיבתיים. ההצעה שלנו אמנם מתמקדת בהשפעת סיבתית של היררכיה על וסוכנות על פערי מידע, אבל דרך בחינת השאלות הסיבתיות אנו יכולים לענות על שאלות היסטוריות חשובות ולתאר באופן דיסקריפטיבי את מערכת הקהילות. באמצעות האמידות שאנו מציעים אנו יכולים לענות בראש ובראשונה על השאלה "איפה הכוח?", האם אלו היו קהילות גדולות ועשירות שהצליחו להטות את המערכת לטובתם, או שאלו היו דווקא קהילות קטנות ועניות? מה הייתה ההשפעה של האצולה על מערכת הקהילות? האם היחסים בין הקהילות היו יחסי מיקוח, או שמא יחסים של שיתוף פעולה? האם ההבדל בין המזרוח והמערב השפיעה על חלוקת הכח?

בנוסף, ולא פחות חשוב, אנחנו יכולים להציג אמידה אמפירית למודל שמשלב נורמות, תרבות, חוקים ומוסדות, אנו מראים השפעות של אכיפה פנימית ואת נקודות החוזק והחולשה שלה. אנו מראים כיצד אלו גורמים שמשפיעים זה על זה. בהצעת המחקר שלנו התמקדנו רק בהיבט של השפעת ההיררכיה על דיווחי הקהילות, אבל, יש חשיבות רבה לכך, שמה שאפשר את האכיפה הפנימית הייתה מסגרת דתית, אכיפה שהתאפשרה אודות לשבועה ולחרם ככלי אכיפה ולחוקים הלכתיים כמערכת שיפוטית.

התרומה המרכזית שלנו למחקר, היא נושא המחקר. לא ניתן להתעלם מהאופן שבו קהילות, סמכות והיררכיה משפיעים על חלוקת המשאבים ומשמשים כסוכנות עבור חברי הקהילה. הקיום של קהילות מחייב שיווי משקל שבו לכל חברי הקהילה יש תמריץ להישאר חברים בה ולכל אחד יש יכולת השפעה על חבר אחר בקהילה, ולכולם יחד על כולם יחד. אנו שואפים לתרום לספרות המחקר על מודלים של תורת המשחקים על השפעה קהילתית, שיתוף פעולה וא-סימטריה באינפורמציה באמצעות הצגת מקרה בוחן שמדגים את האופן שבו קהילה יכולה לבחור לחלק את משאביה, באופן פנימי, באמצעות חוזים פנימיים ואכיפה פנימית.

על אף שבמסגרת הצעת המחקר היה ניתן להציג נתונים 'היפותטיים', בהצעת המחקר שלנו הצגנו רק נתונים שקיימים בוודאות, כך שניתן ליישם את ההצעה בפועל יחסית בקלות. בפרט, אנו מעוניינים להעיר את תשומת הלב על מיעוט המחקר הכלכלי על יהודי מזרח אירופה בתקופה שקדמה לשואה. הספרות על הנושא מתחלקת בין ספרות מודרנית, שמתבצעת בעיקר בפולנית, ספרות בעשורים הראשונים לאחר השואה שהתבצעה בעברית, וספרות לפני המלחמה שהיא בעיקר באידיש. קיים היום מחקר משמעותי על יהודי פולין בפולנית, אך לא קיימים חוקרים ישראלים שעושים זאת. הספר החשוב ביותר אודות המפקד, שהוא גם ככל הנראה המקור היחיד שניתח כלכלית את המפקד, הוא ספרו של מאהלר "די יין אין פולנד אין ליכט פון צייטלוג", מקור זה, כמו רבים אחרים, הולך ונשכח, והידע המחקרי בנושא הופך לנחלת יחידים.

