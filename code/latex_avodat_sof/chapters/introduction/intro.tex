עבודה זו בוחנת מקרה פרטי של בעיות סיכון מוסרי \e{(Moral Hazard)} במסגרת של אכיפה פנימית. 
אנו בוחנים את ההשערה כי ככל שקוקאליציה הופכת יותר היררכית, יעילות האכיפה יורדת עקב בעיות הולכות וגוברות של סיכון מוסרי. ההנחה שלנו היא שפיקוח הדוק יותר באמצעות אכיפה פנימית ברשת היררכית מגדיל את פערי האינפורמציה של הקואליציה לגבי היחידה המפוקחת.

המחקר מנצל קונטקסט היסטורי ייחודי של קואליציה של קהילות יהודיות שהתאגדו למסגרת על-קהלתית בשם "ועד ארבע ארצות" בסוף המאה ה-18 בפולין-ליטא כדי לגבות מיסים מהקהילות היהודיות. ביטולו הפתאומי של הוועד בשנת 1764 ומפקד האוכלוסין שבוצע מיד לאחר מכן מאפשרים לבצע השוואה בין נתוני התשלום של הקהילות לפני ואחרי הביטול, ובכך לאמוד את רמת הא-סימטריה באינפורמציה שהייתה קיימת במסגרת הוועד. המחקר מנצל הבדלים בריבוד ההיררכי של הקהילות במערכת בדי לאמוד את השפעה של ההיררכיה על פערי המידע ומשתמש בשינוי מדיניות משנת 1713, אשר קבע את הכפיפות ההיררכית של יחידות גיאוגרפיות קטנות לפי מרחקן מבית הכנסת של הקהילה הגדולה כמשתנה עזר.

אנו משתמשים בנתונים ממפקד האוכלוסין היהודית בפולין-ליטא שנערך בשנים 1764-1765, מיד לאחר פירוק מערכת הוועד כדי להשוות אותם לנתוני גביית המיסים משנת 1764. ההשערה היא שקהילות בעלות מבנה היררכי מורכב יותר הסתירו יותר מידע בהשוואה לקהילות בעלות מבנה שטוח, וכי תוספת השכבה ההיררכית היא שגרמה להסתרת המידע. ההנחה היא שככל שמספר הרמות ההיררכיות בתוך הוועד גדל, כך יכולת הרשויות לאסוף מידע מדויק על האוכלוסיות החייבות במס ולפקח על תהליך הגבייה פוחתת, מה שמוביל לפער גדול יותר בין המיסים שנגבו בפועל לבין היכולת הכלכלית האמיתית.

התרומה של הצעת המחקר היא בעיקרה הצעת המסגרת אמפירית לבחינת פערי מידע בבעיות ויישום המינוח התיאורטי של בעיות סיכון מוסרי על הקונטקסט, המאמר מציע דרך אמפירית לבחון את השלב שבו אכיפה פנימית הופכת לבלתי יעילה כתלות בריבוד ההיררכי של מנגנון האכיפה.יתר המחקר בנוי באופן הבא: פרק 2 סוקר את הקונטקסט ההיסטוריף פרק 3 סוקר את הספרות ההיסטורית והכלכלית בנושא, פרק 4 מציג את האסטרטגיה האמפירית, פרק 5 מסכם. 
\section{הקדמה}
\subsection{קונטקסט היסטורי}


שורשי הממשל העצמי היהודי בפולין-ליטא נעוצים במאות ה-12 וה-13, כאשר מהגרים יהודים ממערב אירופה החלו להתיישב באדמות פולין. עם גידול האוכלוסייה היהודית, התפתחה רשת של מועצות קהילתיות אוטונומיות, המכונות "קהילות",  שנועדו לנהל את ענייני הפנים של הקהילה היהודית ולשמש כמתווכות בין יהודים לבין מדינת פולין-ליטא. הקהילות היו אחראיות למגוון רחב של תפקידים, ביניהם גביית מיסים, ניהול מערכת המשפט, חינוך דתי ורווחה חברתית.

מאמצע המאה ה-16, הועד החל לשמש כנציג יהודי פולין שאחראי על גביית המס וחלוקתו בין כלל יהודי הכתר. גביית המיסים נעשתה באמצעות חכירת המס - הוועד הציע תשלום מסויים לכתר, הכתר הסכים להחכיר את המס תמורת התשלום, והוועד גבה את המס וחילק אותו כראות עיניו, חלק מהמיסים הועברו לטובת השלטון והשאר למערכת הקהילתית. מס גולגולת הייתה שיטת המיסוי נפוצה בימי הביניים וחכירת מיסים הייתה גם היא נוהג מקובל. היחודיות של ועד הקהילות הפולני היא בכך שהוא אפשר גביית מיסים קולקטיבית על-לאומית בפולין, ליטא, אוקראינה ובלארוס, ובשל האוטונמיה המשמעותית של הועד.

במערכת הפולנית, מחוזות או "וויוודויות" נשלטו לעתים קרובות על ידי המגנאטים, בעוד שעיירות וכפרים היו תחת סמכותה של האצולה הזוטרה. הקהילות היהודיות שיקפו את חלוקת הכוח: ועד הקהילות הורכב מנציגים של "ועדי גלילות" המקבילים לויווידות, ועדי הגלילות הורכבו מ"קהילות ראשיות" שכל אחת מהן שלטה על קהילות עירוניות, הקהילות העירוניות שלטו על "סביבות" שהורכבו מכפרים ומפריפריה עירונית.
מאז 1717, לאחר התחזקות המגנאטים, נקבע כי וועד הקהילות לא משמש כגוף שגובה את המס, אלא רק כגוף שמעריך את גובה נטל המס של כל קהילה. מס הגולגולת היהודי היה חלק ממערך מיסים בשם "היבֶּרְנַה" שנועד לכיסוי הוצאות הצבא. מתוך חשש לניצול לרעה של המערכת ההיררכית כך שקהילות מסויימות ישלמו חלק משמעותי יותר בהכנסות, הסיים קבע כי הערכת מיסי הגולגולת צריכה להיות "הערכה צודקת", כלומר, כזו שמציינת באופן מדוייק, ולכל הפחות פרופוציונאלי, את מספר האנשים בקהילה, נציגי הקהילות היו צריכים להישבע על כך. 

קיים קונצנזוס נרחב בנוגע לקשר  בין הקמת "ועד ד ארצות" בפולין ו"ועד המדינה" בליטא לבין גביית מס הגולגולת. חוקרים רבים, כגון מסכימים כי ביטול מוסדות אלה נבע מאי-אפקטיביות של המוסדות בגביית המס. עם זאת, קיימת מחלוקת לגבי הסיבות שהובילו לירידה באפקטיביות של המערכת בגביית המיסים.  חוקרים רבים מתמקדים בגורמים חברתיים ששחקו לאורך זמן את השוויון בנטל המס בין הקהילות דינור למשל, טען כי ניתן לראות התפרקות הקהילות את ראשיתה של תנועת החסידות שנולדה כביטוי למלחמת מעמדות ול'מרד' עממי שהביא לקריסה. חלק מהחוקרים מייחסים את הקריסה לקונפליקטים הפנימיים בקרב האליטה הפולנית או בקרב אליטה יהודית. \cite{teller_early_2010-1} ו\cite{kazmierczyk_permanent_2022}. אחרים מתארים יחסי מיקוח בין האליטה לעם, כאשר הכוח עבר בהדרגה מאליטה מצומצמת לחוכרים פרטיים ולקהילות קטנות שהיוו קבוצות לחץ.  \cite{kalik_scepter_2009}
אחרים מתמקדים מתמקדת בגורמים חיצוניים בלתי תלויים בחלוקת הנטל כמו מרד הקוזקים של 1648 שיצר שחיקה הדרגתית ביכולת הפיננסית של המערכת, או שינויים תרבותיים אקסוגניים כגון עליית התנועה הקבלית.\cite{hundert_jews_2004}

מערכת זו פעלה עד 1764 ואופיינה בחוסר יציבות מתמיד ובתלונות מצד יחידות כלכליות קטנות על מיסוי יתר, \citep{kazmierczyk_permanent_2022}  כינה את התקופה הזו "משבר מתמיד" בשל מצבים חוזרים ונשנים של 'מרד' של קהילות. ב-1764 לאחר התחזקות השלטון המרכזי וביטול זכות הווטו של המגנאטים, החליט השלטון המרכזי לבטל את ועדי הקהילות בפולין ובליטא ולעבור לגביה ישירה מהקהילות ללא תיווך של הוועד. מיד לאחר ההחלטה על ביטול מנגנון הוועד, נערך מפקד אוכלוסין בקרב כלל היהודים בתחומי האיחוד
המפקד חשף פערים משמעותיים בין האוכלוסייה היהודית בפועל לבין הנתונים שדווחו על ידי הקהילות, מה שהצביע על דיווח חסר נרחב והעלמות מס.

המחקר הכלכלי בנוגע לשיתוף פעולה באמצעות קואליציה ואכיפה פנימית ענף למדי, למשל, ההסדרים השיתופיים בין הסוחרים המגריביים וסוכניהם מעבר לים \e{\cite{greif_contract_1993}} ונורמות האכיפה הפנימית בין חוואים ובעלי קרקעות \e{\cite{ellickson_order_1991}}. לעומת זאת, המחקר על התלות בין אכיפה חיצונית ואכיפה פנימית הוא בעיקר תיאורטי.
 אסמוגלו \cy{acemoglu_sustaining_2020}  הציג מסגרת שבה 'אוכפים פנימיים' בוחרים את מידת האכיפה לפי הרווח שלהם מהאכיפה ושיווי המשקל נשמר רק כאשר התמורה לאכיפה מספיק גדולה.


\section{אסטרטגיה אמפירית}

\subsection{מכניזם תיאורטי}

הטיעון התיאורטי שלנו מבוסס על מסגרת דומה שבה בעיית "סיכון מוסרי" במערכת היררכית, שבה היכולת של סוכנים לנצל את המערכת לטובתם גדולה יותר בהשוואה למערכת "שטוחה", והשליטה באינפורמציה קלה יותר. במסגרת שלנו, כל קהילה היא 'אוכף' מלבד הקהילה האחרונה בשרשרת ההיררכית, וההשערה  היא שקהילות שאופיינו בריבוד היררכי מורכב יותר נטו ליצור פערי אינפורמציה גדולים יותר. ככל שהמערכת הופכת היררכית יותר, האפקטיביות של האכיפה פוחתת, בעיית ה-\e{Moral Hazard} גדלה והמידע של המנהלים מצטמצם.

התמריץ של הקהילות הוא לשמר את המערכת כדי להבטיח חבות מס נמוכה יותר באופן קולקטיבי ולהשיג שירותים מסוימים מהשלטון. עם זאת, ככל ששחקן קטן יותר, יש לו תמריץ גדול יותר לרמות ולהסתיר את חבותו במס, שכן במקרה של קריסת המערכת הוא יפסיד פחות. בעוד שניתן לאכוף סנקציות נגד יחידים שמרמים, קשה יותר לעשות זאת נגד קבוצה גדולה מחשש למרד והתנגדות. הניתוח שלנו מציע מבחן אמפירי לגבי יעילות האכיפה פנימית במסגרת של א-סימטריה באינפורמציה תחת קואליציה עם היררכיה, כאשר, אנו משתמשים בנתונים של המפקד כמדד לא-סימטריה באינפורמציה שנגרמת על ידי הקואליציה. 

בהתאם לממצאים של קליק המתארים "התפוררות" של המערכת ההיררכית. אנו טוענים כי ככל שיש יותר שכבות היררכיות, המס המדווח יהיה נמוך יותר, על אף היחלשות עוצמת ההשפעה ככל שמתרחקים ממרכז הכח בהתאם למודל של אסמוגלו. בנוסף, אנו שולטים על שיעור המס ששולם על ידי הקהילה מכלל חבריה, בהתאם לתיאוריה של אסמוגלו הטוענת שיעילות האכיפה עולה ככל שלאוכפים יש תמריץ לאכוף.

אנו מנצלים את העובדה שהמערכת ההיררכית של הקהילות לא הייתה זהה, חלק מהקהילות הראשיות קיבלו ייצוג אוטונמי כישות כלכלית עצמאית שאינה כפופה ליחידה אחרת, בעוד שקהילות ראשיות אחרות היו כפופות לקהל איזורי ('גליל'),  החלוקה ל'גלילות' מקבילה למערכת פדרלית שבה ערים מרכיבות גוף כמו פרובינציה או state והגוף מייצג אותם במסגרת פרלמנטרית רחבה יותר וייצוג 'אוטונמי' מקביל לנציגות ישירה של העיר או המחוז, ללא תוספת השכבה ההיררכית. 

החלוקה לגלילים התחילה כחלוקה גיאוגראפית-אדמינסטרטיבית שבה נטל המס חולק מטעמי נוחות בין קהילות, אך אך בשל המשמעות שהייתה לחלוקה הזו במונחי שותפות בנטל המס, חלוקה ברווחי חכירה, קבלת צ'ארטרים ופריבלגיות מסחר, קהילות מסויימות בחרו להתאגד ולהרכיב ישות כלכלית עצמאית שאינה כפופה לקהל איזורי ומייוצגת ישירות בוועד המרכזי. "ועד 4 ארצות" הורכב למעשה-22 ישויות, כ-10 קהילות עם ייצוג מחוזי, ו-7 קהילות עם ייצוג אוטונמי, מספר הערים המרכיבות את הקהל היה דומה בכל אחת מהקהילות. קהילות הכפופות ביותר היו עיירות הכפופות לקהל עירוני, הכפוף ל'קהילה ראשית' שמיוצג במסגרת 'גליל'. ביניהם,  המשמעות היא, שחלק מהקהילות הראשיות היו נתונות למערכת עם שכבה היררכית נוספת, וחלקן יוצגו באופן ישיר.

הנחה המרכזית של האמידה האמפירית שלנו היא שמפקד האוכלוסין משקף זעזוע אקסוגני לא-סימטריה באינפורמציה בין השלטון לקהילות, זעזוע שניתן לאמוד באמצעותו את ההפרש בין האוכלוסיה האמיתית לאוכלוסיה המדווחת. אנו לא מניחים שהמפקד מדוייק או שאינו מושפע מהמורכבות ההיררכית, אלא  שההפרש בין תשלום המיסים בשנת 1764 ובין המפקד של 1764 משקף עבור כל קהילה בקואליציה, את מידת היכולת שלה להסתיר את מספר הנפשות שלה בכפוף למערכת ההיררכית. תחת הנחה זו אנו משערים שעצמת הא-סימטריה באינפורמציה גדולה יותר ככל שהריבוד ההיררכי גדול יותר. ניתן לנמק את ההשערה כעלות קבועה של פיקוח חיצוני שעולה עם כל שכבה היררכית נוספת עקב הצורך לפקח על יחידה כלכלית נוספת ועליה בעלות האינפורמציה עקב מורכבות המערכת, לחלופין, ניתן להתמקד בעלות קבועה שמקורה בסוכן, כאשר מקור ההפרש הוא בכך שהעלות מתחלקת על פני מספר קטן יותר של מפוקחים. האסטרטגיה האמפירית שלנו מנצלת את העובדה שהמערכת ההיררכית התבטלה לפני מפקד האוכלוסין, כמעט מיד לאחר תשלום מס הגולגולת באותה שנה. זה מאפשר לנו להפריד בין אסימטריה באינפורמציה שנבעה מהמערכת ההיררכית לבין העלויות ה'טבעיות' של הסוכן.


\subsection{נתונים}
ראשית, אנו מנצלים את נתוני מפקד האוכלסין ונתוני מס הגולגולת כדי לבנות את המדד המרכזי שלנו לאי-סימטריה במידע על ידי השוואת נתוני האוכלוסייה המדווחים עם האוכלוסייה בפועל כפי שתועדה במפקד.
כדי לבנות את המשתנה התלוי שלנו, אנו משתמשים בנתונים לגבי תשלום מס הגולגולת ב-1764, לצורך כך, אנו משתמשים ברישומי מס הגולגולת עבור השנים 1717-1764 שהתגלו על קליק (2009) בארכיון האוצר הפולני. אנו מנצלים את העובדה שהמס עמד על שני זלוטי לנפש כך שיהיה ניתן להשתמש בהם כדי להשוואת האוכלוסיה בפועל לאוכלוסיה המדווחת.
לאחר מכן, אנו משתמשים בנתוני מפקד האוכלוסין היהודי בפולין-ליטא משנת 1764-1765 כבסיס להשוואה לגבי פוטנציאל גביה ללא המערכת ההיררכית. יחד עם נתוני מס הגולגות של אותה שנה אנו יכולים לחשב מדד א-סימטריה עבור כל קהילה ולהעריך את היקף הדיווח החסר של מערכת הקהילות.

אנו משתמשים במסד הנתונים של "דתות וזרמים דתיים בכתר הפולני במחצית השנייה של המאה ה-18"  (RCP) \cite{szady_geografia_2023} כדי לייצר משתנה עזר של מרחק מבית הכנסת הקרוב ביותר בשנת 1713. מסד הנתונים מורכב מנתונים גיאו-מרחביים, על כל מוסדות דת שהתקיימו במחצית השניה של המאה ה-18 ומספק נתונים על כ-1600 יחידות גיאוגרפיות שבהן התקיים בית כנסת, כאשר מתוכן, 30 בתי כנסת שייכים ל"קהילה ראשית".

לבסוף, אנו משתמשים בנתוני AtlasFontium \citey{historicalatlaspoland} על הכתר הפולני ליטאי המאה ה-18, כדי ליצור משתני בקרה אודות החלוקה האדמינטרטיבית של פולין לפני 1772 (חלוקת פולין) לפי פרובינציה, עיר ועיירה. 

מדד ההירכיה שלנו מנצל את ההבדלים בריבוד ההיררכי בוועד הקהילות, ראשית, אנו יוצרים משתנה פשוט למיקום היררכי המוגדר כסכום של מספר היחידות הממונות על יחידה כלכלית מסויימת עבור כל איזור. שנית, אנו מנצלים אנו מנצלים את העובדה שהקהילות השונות היו כפופות למערכת היררכית שונה כדי ליצור מדד להיררכיה 'שטוחה' לעומת היררכיה מורכבת. 

מדד המיקום ההיררכי H, מסמן עבור כל יחידה גיאוגראפית את מספר האוכפים הממונים עליה, כאשר יחידה, עיר, עיירה או כפר עם K יחידות ממונות תקבל ציון של K+1.
המדד לריבוד היררכי, מציין את מספר האוכפים המקסימלי האפשרי במסגרת ההיררכית וקבוע בין כל היחידות הנמצאות באותה מסגרת היררכית כאוכפים או ככפופים.
למשל, קהל של עיירה הכפוף לקהל עירוני הכפוף למחוז (קהילה ראשית) הכפוף לפרובינציה (גליל) שאמון על גביית מיסים מכפר מסויים, כפוף ל-3 אוכפים וממונה על נאכף 1, לכן, המדד למיקום ההיררכי יהיה $H=4$, והמדד לריבוד ההיררכי יהיה $HS=5$.

לאחר מכן, אנו מסווגים את היחידות לפי הרמה ההיררכית המקסימלית במבנה: כפיפות מחוזית, כפיפות עירונית, וייצוג ישיר, ויוצרים משתנה דמי HS השווה ל-1 אן רמת ההיררכיה המקסימלית $HS<=3$ ושווה ל-0 אם $H<3$. הסיבה לבחירה של $H<3$ כסף לרמת ההיררכיה היא שזהו הסף המבדיל בין קהילה במערכת 'אוטונמית' לקהילה במערכת 'איזורית'. 


\begin{table}[h]
    \centering
    \renewcommand{\arraystretch}{1.4}
    \begin{tabular}{|c| c| c |c|>{\centering\arraybackslash}m{4cm}|>{\centering\arraybackslash}m{4cm}|>{\centering\arraybackslash}m{4cm}|}
   \hline
    \textbf{מיקום היררכי - H} & \multicolumn{3}{c|}{\textbf{ריבוד היררכי - HS}}  \\ \hline
     & 4 & 3 & 2 \\    \hline
    1 & קהילה ראשית & מערכת אוטונומית & קהילה חוץ-טריטוריאלית \\
    
    2 & קהילה איזורית (גליל) & קהילה עירונית & קהילה כפופה \\
    
    3 & קהילה עירונית & קהילה כפופה & \\
    
    4 & קהילה כפופה & & \\
    \hline
    HS & \multicolumn{2}{c}{0} & 1 \\ \hline
\end{tabular}
    \end{table}
        
\subsection{אמידה אמפירית}

ראשית אנו מציגים מודל בסיס של השפעת המיקום ההיררכי על הדיווח במפקד האוכלוסין של 1764, כאשר דיווח המיסים של 1764 משמש עבורנו כמשתנה בקרה שמתאר את האוכלוסייה המדווחת. כפי שמתואר בהרחבה בנספח (1)  משתנה ההיררכיה מאפשר לנו לאמוד את ההשפעה של ההיררכיה על האוכלוסייה הלא-מדווחת ולכן מציג מדד לפערי אינפורמציה. נתוני מפקד האוכלוסין משמשים כפרוקסי למספר האוכלוסייה האמיתי, אך אנו לא מניחים שדיווחי מפקד האוכלוסין מדויקים אלא שהם משקפים מידע המתקבל תחת אכיפה לאחר תשלום עלויות האינפורמציה. לכן, ניתן לראות את השפעת ההיררכיה כ"עלויות סוכן" במקרה שהיררכיה גבוהה יותר מתואמת עם דיווחים גבוהים יותר, וכ"פרמיית סוכן" במקרה שההיררכיה מתואמת עם דיווחים נמוכים יותר.

אנו רוצים לאמוד את ההשפעה של הוספת שכבה היררכית, כלומר, מעבר ממערכת 'מורכבת' לעומת מערכת 'שטוחה', לשם כך, אנחנו מנצלים את ההבדל בריבוד ההיררכי בתוך הקהילות היהודיות. המשתנה המסביר שלנו הוא האינטראקציה בין המשתנה שמתאר את מספר האוכפים הממונים על היחידה הגיאוגרפית לבין משתני הדמי של מעבר למערכת "אזורית". 
המשוואה שלנו היא:
\begin{align}
    PollTax_{id} = \beta_0 + \beta_1 H_i + \beta_2 Cen_i + \beta_3 HS_i + \beta_4 H_i*HS_d + \mu_{id}
\end{align}

במשוואה זו, $PollTax^{1764}_i$ מתאר את תשלומי מס הגולגולת של שנת 1764 לפי דיווח קהילתי לשלטון על ידי קהילה i עם קהילה ראשית d, $Census^{1764}_i$ הוא מפקד האוכלוסין של שנת 1764 ומשמש כמדד לאוכלוסייה האמיתית, ו-$H$ הוא משתנה קטגוריאלי מ-1 עד 4 לפי מספר היחידות הממונות על יחידה גיאוגרפית.

במשוואה זו, $\beta_1$ מייצג את ההשפעה של המיקום ההיררכי על דיווח המיסים של הקהילה בשנת 1764, $\beta_2$ מייצג את שיעור מדווחי המיסים בשנת 1764 שלא הושפעו מהמבנה ההיררכי, $\beta_3$ מייצג את ההשפעה של מעבר ממערכת היררכית אחת לאחרת על אי-דיווח, ו-$\beta_4$ הוא המקדם המעניין שמציין את ההשפעה של מעבר מיחידה מרובדת היררכית ליחידה לא מרובדת היררכית.

מודל זה מאפשר לנו לנצל את ההבדל במבנה ההיררכי של הקהילות כניסוי טבעי, כאשר ההבדל בין צורות השלטון משמש לבניית קבוצות הטיפול והבקרה. ה"טיפול" מוגדר כמעבר של יחידה גיאוגרפית ממבנה היררכי 'אוטונומי' למבנה היררכי 'מורכב'. קבוצת הטיפול היא הקהילות שהיו נתונות למערכת האזורית, וקבוצת הבקרה שלנו הן הקהילות האוטונומיות שהציגו ריבוד היררכי נמוך יותר והיו חשופות לפחות תמריצים להסתרת מידע על ידי גורמים מתווכים.

הנחת הזיהוי במודל זה היא שאין גורמים שהשפיעו על הדיווח במפקד האוכלוסין ומתואמים שיטתית עם הריבוד ההיררכי של המערכת. עם זאת, קשה להצדיק הנחה כזו, מכיוון שלא מדובר בניסוי טבעי טהור, וייתכן שהיו גורמים חיצוניים שהשפיעו על מידת שיתוף הפעולה של הקהילות. ראשית, ייתכן שגורמים חיצוניים בעלי יכולת השפעה על הרמה ההיררכית של הקהילה יכלו ליצור פערי אינפורמציה. אצילים התערבו לא אחת בהחלטות של ועד הארצות והייתה להם יכולת השפעה על רמת הדיווח והציות של הקהילה. כיוון הסיבתיות בין מבנה היררכי לא-סימטריה באינפורמציה עשוי להיות דו-כיווני, ייתכן שקהילות שנטו להסתיר מידע, כמו קהילות עשירות, אימצו היררכיה מורכבת יותר כאמצעי בקרה לזרימת מידע ושליטה.


%%%%%%%%%%%%%%%%%%%%
\subsection{משתנה עזר}

כדי להשוות את ההבדלים בפערי האינפורמציה המושפעים רק מריבוד היררכי, אנו מנצלים שינוי מדיניות שנעשה על ידי וועד הארצות ב-1713, מדיניות זו קבעה כי כל קהילה שאין בתחומה בית כנסת ונמצאת במרחק של עד 2 מייל מקהילה ראשית, שייכת לקהילה זו. המרחק נקבע באופן שרירותי, תחת עקרונות הלכתיים של 'תחום שבת' ולכן אנו יכולים להשתמש במדיניות הזו כמשתנה עזר, שמאפשר לנו לבודד את ההשפעה של גורמים אנדוגניים ולאמוד רק את ההשפעה של הטיפול, ולפרש את האומד כגורם סיבתי.  

אנו משתמשים בנתוני RCP כדי ליצור משתנה דמי המציין האם קיים בית כנסת בקהילה, ומשתנה רציף המציין את המרחק לבית הכנסת הקרוב ביותר. לאחר מכן אנו יוצרים משתנה דמי שמאפשר טווח מרחק מסוים 'חלון' השווה לאחד עבור מרחק הגדול מ-1000 מטר וקטן מ-3000 מטר ו-0 אחרצ. הנחת הזיהוי שלנו היא שלא קיים גורם נוסף המשפיע על האינטראקציה בין מרחק מבית הכנסת והרמה ההיררכית ובין פערי האינפורמציה, שלא דרך הריבוד ההיררכי.


אנו מנצלים שינוי מדיניות שנעשה על ידי ועד ארבע ארצות בשנת 1713. מדיניות זו קבעה כי כל קהילה שאין בתחומה בית כנסת ונמצאת במרחק של עד 2 מייל מקהילה ראשית, תשתייך לאותה קהילה ראשית. הקריטריון של 2 מייל נקבע באופן שרירותי, על בסיס העקרון ההלכתי של 'תחום שבת', מה שיוצר יוצר חוסר רציפות טבעי. חיתוך זה מאפשר לנו להשתמש ברגרסיית אי רציפות (RDD) להערכת ההשפעה של המבנה ההיררכי על פערי המידע, תוך התייחסות לנקודת החיתוך כגורם אקסוגני. ראשית, אנו בודקים האם קיימת אי רציפות מובהקת בגבול תחום השבת בהקצאת הקהילות לקהילות ראשיות:


\begin{align}
H_{i} = \alpha_0 + \alpha_1 \mathbf{1}{\text{Distance}_i \leq 2} + \alpha_2 \text{Distance}_i + \alpha_3 \text{Distance}_i \times \mathbf{1}{\text{Distance}_i \leq 2} + \varepsilon_i
\end{align}

כאשר $H_i$ היא הרמה ההיררכית של קהילה i בשנת 1764, ו-$\text{Distance}_i$ הוא המרחק של הקהילה מבית הכנסת הקרוב ביותר ב-1713. מקדם מובהק $\alpha_1$ יעיד על קיומה של אי רציפות ברמה ההיררכית בגבול ה-2 מייל.

בהינתן מציאת אי רציפות מובהקת, נוכל להשתמש במרחק עד 2 מייל כמשתנה עזר (IV) להערכת ההשפעה הסיבתית של הרמה ההיררכית על פערי המידע. משוואת השלב הראשון נתונה על ידי:
\begin{align}
H_i \times HS_{d} = \gamma_0 + \gamma_1 \mathbf{1}{\text{Distance}_i \leq 2} + \gamma_2 \text{Distance}_i + \gamma_3 \text{Distance}_i \times \mathbf{1}{\text{Distance}_i \leq 2} + \nu_i
\end{align}
הערכים החזויים של המודל, מייצגים את הפוטנציאל של איזור גיאוגראפי להיות מושפע מהריבוד ההיררכי (למעשה, מהאינטראקציה), כאשר, איזורים רחוקים יותר הם בעלי פוטנציאל נמוך יותר. הפוטנציאל הוא גורם אקסוגני אם אכן קיימת אי רציפות ואם הדבר היחיד שמשתנה בנקודת הרציפות הוא תחום שבת. אם נמצא כי $\alpha_1$ מובהק, משמע שיש אי רציפות בגבול ה-2 מייל. במקרה כזה, נוכל להשתמש במרחק מתחת ל-2 מייל כמשתנה עזר להשפעת המבנה ההיררכי על פערי המידע. אנו מעריכים את ההשפעה באמצעות מודל 2SLS:

שלב 2: 
\begin{align}
InfoGap_{i} = \beta_0 + \beta_1 \widehat{H_{i} \times HS_{d}} + \beta_2 \text{Distance}_i + \epsilon_i
\end{align}

כאשר $InfoGap_{i}$ הוא מדד לפער המידע של קהילה iבשנת 1764, ו-$\widehat{H_{i} \times HS_{d}}$ הם הערכים החזויים מהשלב הראשון.

המודל מניח כי גבול תחום שבת מקהילה ראשית לא מתואם שיטתית עם גורמים אחרים שמשפיעים על פערי אינפורמציה שלא דרך היררכיה. הנחה זו עשויה להיות בעייתית אם, למשל, המרחק מהגבול אפשר לקהילות להסתיר יותר מידע. הבעיה המשמעותית יותר היא שמודל ה-RDD מעריך את ההשפעה המקומית סביב סף ה-2 מייל, ולא מן הנמנע שהאפקט של הריבוד ההיררכי בקרב הקהילות הקרובות יהיה שונה משמעותית מזה של קהילות רחוקות, כך שהתוצאות שנקבל באמצעות האמידה הזו לא מספקת מספיק אינפורמציה לגבי ההשפעה של הריבוד ההיררכי על פערי מידע בקרב כלל הקהילות.

הטווח שאנחנו מאפשרים נבחר באופן שרירותי כמרחק סביר שמאפשר לתפוס את ההשפעה של המדיניות, תוך הקטנת החשש לאקסוגניות. ככל שנגדיל את הטווח שאנחנו מאפשרים יהיה קשה יותר להצדיק את הנחת האקסוגניות, ככל שנצמצם אותו, יהיה קשה להראות רלוונטיות. כדי לבחון את רגישות התוצאות שלנו אנו בוחנים את הרגישות על סדרה של טווחים, במרחק של 500,1000,1500 מהגבול של 'תחום שבת'. אנו מצפים שההשפעה תלך ותפחת, במונחי השפעה על המקדם או במונחי מובהקות סטטיסטית אבל שהשינויים יהיו קונסיסטנטיים. 

כדי לבחון אם התוצאות אכן משפיעות דרך המערכת ההיררכית הקהילתית, ולא דרך גורמים אדמיניסטרטיביים, אנו מריצים את הרגרסיה של משתנה העזר על קהילות של יהודים קראים שגם הם דווחו במפקד, קהילות אלו לא היו כפופות לחוק ההלכתי ולא נכללו במערכת הקהילתית, אבל החוק הפולני כלל אותם במסגרת החלוקה האדמיניסטרטיבית היהודית. במסגרת הצעת המחקר אנו לא יכולים לומר כמה קהילות כאלו התקיימו, ועבור כמה יש נתונים, אולם קיימות קהילות גדולות (למשל, טרוקי) שעבורן יש נתונים, אנו מצפים שעבור קהילות אלו התוצאות יראו משתנה לא מובהק.


כדי לבחון האם התוצאות רגישות לבחינת משתנה העזר שלנו וכדי לבחון האם התוצאות שלנו קונסיסטנטיות ואינן מונעות על ידי משתנה מושמט המתואם עם המרחק מתחום שבת, אנו מציעים משתנה עזר אחר, הבוחן את החלוקות במערכת האדמיניסטרטיבית הפולנית, שהתבצעו עקב ירושה, מלחמות, או רכישות של אצילים. אנו משתמשים בנתונים של האטלס ההיסטורי של פולין המכיל מידע על בעלות על כפרים וערים עבור כל המאה ה-18, ומאפשר לנו לאתר מקרים של שינוי בעלות על יחידות. אנו מאתרים את כל המקרים בהם התבצע שינוי בעלות בין השנים 1717-1764 ומריצים אותו על משתנה האינטראקציה.

המעבר של יישוב מבעלות של אציל אחד לאחר עשוי היה להשפיע על השתייכותה של הקהילה היהודית בתוך המערכת המנהלית. כדי לבחון השפעה זו, אנו יוצרים משתנה דמי באמצעות הנתונים של קליק, השווה ל-1 אם יישוב עבר שינוי בעלות, ו-0 אחרת. לאחר מכן, אנו מריצים רגרסיה של משתנה העזר על משתנה ההיררכיה שלנו, וכוללים אותו במודל (1). הערכים החזויים מהמשוואה מייצגים את הפוטנציאל לשינוי בעלות ולכן אנחנו יכולים להשתמש בהם במודל (1) כגורם אקסוגני:

\begin{align}
H*HS= \alpha_0 + \alpha_1 NobChange +\alpha2 ComChange +\alpha3 NobChange * ComChange + \mu
\end{align}

כאשר $H$ מייצג את משתנה ההיררכיה של הקהילה בשנת 1764, ו-$HS$ הוא משתנה ההיררכיה של הקהילה בשנת 1764. $NobChange$ הוא משתנה דמי המציין שינוי בעלות על יישוב, ו-$ComChange$ הוא משתנה דמי המציין שינוי בשייכות הקהילתית. $\alpha_0$ מייצג את האפקטים הקבועים של כל רמת היררכיה, $\alpha_1$ מייצג את ההשפעה של שינוי במבנה האצולה על הרמה ההיררכית, $\alpha_2$ מייצג את ההשפעה הממוצעת של שינוי קהילתי על רמת ההיררכיה, ו-$\alpha_3$ מייצג יישובים שחוו שינוי בשייכות הקהילתית עקב שינוי בבעלות על הקרקע. $\mu$ מייצג את הטעות האקראית.

ההנחה של המודל היא ששינויי בעלות על יישובים היא לא מתואמים עם פערי המידע בקהילות היהודיות. כלומר, אנו מניחים ששינוי הבעלות על היישובים, שנבעו מאירועים כמו ירושות, מלחמות או רכישות של אצילים, לא היו תלויים בגורמים שקשורים לאופן שבו הקהילות היהודיות דיווחו על המפקד.  עם זאת, חשוב לציין שהנחה זו עשויה להיות בעייתית אם, למשל, הייתה הטיה של אצילים מסוימים לרכוש או להשתלט דווקא על יישובים עם קהילות יהודיות שנטו להסתיר מידע מהשלטון. במקרה כזה, ההנחה לא תתקיים והמשתנה לא יהיה תקף.




\subsection{דיון ומסקנות}
הצעת המחקר הנוכחית בוחנת את הקשר בין מורכבות היררכית של מערכת אכיפה פנימית ליעילות האכיפה, תוך שימוש בהקשר ההיסטורי של ועד ארבע ארצות בפולין-ליטא במאה ה-18. החולשה המרכזית של העבודה היא בכך שאף אחד מהחלקים, התיאורטי או האמפירי, לא עומד מספיק טוב לכשעצמו, המכניזם התיאורטי לא פותח בצורה מלאה והחלק האמפירי קשה למימוש. נתוני המרחק מבית הכנסת ככל הנראה לא יספקו מספיק שונות, וקשה לממש את הרעיון בפועל. נתוני שינוי הבעלות על אדמות אצולה מוגבלים רק למחוז מסוים בפולין, ואין כרגע נתונים על כל שאר המדינות.

עם זאת, הצעת המחקר מספקת תובנות חשובות, מפת דרכים לביצוע של המחקר בעתיד. מבחינה תיאורטית, ההצעה מזהה את חשיבות הא-סימטריה באינפורמציה כגורם סיבתי חשוב ביעילות האכיפה הפנימית בקואליציות. עדויות נרטיביות רבות מראות כיצד הקהילות פעלו להסתרת מידע או תמרון נתונים, אם פערי אינפורמציה מסבירים את הדינמיקה קואליציונית בקונטקסט של ועד הקהילות, הרי שמדובר במקרה של אכיפה קהלתית שהתאפשרה תוך שימוש באמצעי הענשה פיזיים מינימליים לאורך מאתיים שנה על פני איזור המשתרע על פני שטח גיאוגראפי רחב.
מבחינה מתודולוגית, אנו מציעים מפת דרכים לאופן שבו מחקר עתידי יוכל לנצל את השימוש בשונות הדתית. משתנה העזר הראשון שלנו התבסס על אילוץ דתי אקסוגני ואנו מדגימים באמצעותו כיצד אילוצים דתיים יכולים לייצר שונות אקסוגנית של השפעה על אינטראקציה כלכלית דרך מוסדות חוקיים. 
לא פחות חשוב, אנו מעירים את תשומת הלב לגבי נתונים חדשים ומגמות מחקר חדשות שעוסקות בכתר הפולני וביהודי פולין החל מהעת החדשה המוקדמת. חלק מהנתונים הם עבודה שהתפרסמה מהחודשים האחרונים, חלק מהנתונים מציגים מידע עשיר מאוד על התנהלות כלכלית, כולל, תשלום מיסים וחכירה לאורך כל המאה ה-16 ולא נעשה בהם עדיין שימוש מחקרי. ניסינו להדגים שימוש בהם, אך אין ספק שמחקר עתידי יוכל לנצל אותם בצורה טובה יותר. 
לבסוף, המחסור המרכזי בעבודה הוא העדר ההתייחסות לפנקסי הקהילות, פנקסי הקהילות הם מקור משלים לנתונים רשמיים של הכתר, פנקסים אלו מכילים רישומים מפורטים על הוצאות הקהילה, חלוקת הרווחים, הוצאות שוחד וחיובי המס. לא השתמשנו בהם מפני שעדיין לא ברור מה היקף הנתונים המלא שניתן להוציא מהם והאם ניתן לעשות בהם שימוש בניתוחים אקונומטריים. אף על פי כן, אין ספק שמדובר במכרה זהב למחקר כלכלי שממתין בקרן זווית.

לסיכום, הצעת המחקר, בעיקרה, היא הצעה להשתמש בקונטקסט ההיסטורי של ועד הקהילות האוטונמי מאז הקמתו ועד קריסתו, כדי לנתח הבטים תיאורטיים ואמפיריים של אכיפה קהלתית.

3873 מילים