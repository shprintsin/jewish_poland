\title{ עלויות אינפורמציה ואכיפה פנימית: המקרה של "ועד ארבע ארצות" בפולין-ליטא במאה ה-18}
\author{שניאור שפרינצין, סם קליינמן, איתי בוצין}
\date{2021}
  \maketitle
\begin{abstract}
  הצעת המחקר בוחנת את השפעת הריבוד ההיררכי על יעילות האכיפה במוסדות אכיפה פנימית, תוך התמקדות במקרה של "ועד ארבע ארצות" בפולין-ליטא במאה ה-18. המחקר מנצל את ביטולו הפתאומי של הוועד ב-1764 ואת מפקד האוכלוסין שבוצע מיד לאחר מכן כדי להעריך את פערי האינפורמציה שהיו קיימים במסגרת הוועד, על ידי השוואת נתוני מס-גולגולת עם נתוני האוכלוסייה בפועל.
  ההשערה המרכזית היא שככל שהקואליציה הופכת יותר היררכית, יעילות האכיפה יורדת עקב עלייה בבעיות סיכון מוסרי. \end{abstract}

% \noindent%
% {\it Keywords:}  3 to 6 keywords, that do not appear in the title
% \vfill
