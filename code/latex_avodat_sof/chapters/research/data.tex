\subsection{מכניזם}
\section{מכניזם}

המחקר הכלכלי בנוגע לשיתוף פעולה באמצעות קואליציה ואכיפה פנימית ענף למדי, למשל, ההסדרים השיתופיים בין הסוחרים המגריביים וסוכניהם מעבר לים \cite{greif_contract_1993} ונורמות האכיפה הפנימית בין חוואים ובעלי קרקעות \e{\cite{ellickson_order_1991}}. לעומת זאת, המחקר על התלות בין אכיפה חיצונית ואכיפה פנימית הוא בעיקר תיאורטי. הניתוח שלנו מציע מבחן אמפירי לגבי 
היעילות של אכיפה פנימית במסגרת של א-סימטריה באינפורמציה תחת קואליציה עם היררכיה, כאשר, אנו משתמשים בנתונים של המפקד כמדד לא-סימטריה באינפורמציה שנגרמת על ידי הקואליציה.

הטיעון התיאורטי שלנו מבוסס על בעיית "סיכון מוסרי" \e{(Moral Hazard)} במערכת היררכית, שבה היכולת של סוכנים לנצל את המערכת לטובתם גדולה יותר בהשוואה למערכת "שטוחה", והשליטה באינפורמציה קלה יותר. אנו טוענים כי ככל שהמערכת הופכת היררכית יותר, האפקטיביות של האכיפה פוחתת, בעיית ה-\e{Moral Hazard} גדלה, והמידע של המנהלים מצטמצם עקב שכבות נוספות של פיקוח.  
בנוסף, קיימים יחסי מיקוח בין האוכפים הפנימיים וחברי הקואליציה,  בדומה למסגרת התיאורטית שהוצגה על ידי \cy{acemoglu_sustaining_2020}  שבו 'אוכפים פנימיים' בוחרים את מידת האכיפה לפי הרווח שלהם מהאכיפה. במסגרת שלנו, כל קהילה היא 'אוכף' מלבד הקהילה האחרונה בשרשרת ההיררכית, וההשערה שלנו היא שקהילות שאופיינו בריבוד היררכי מורכב יותר, יצרו ליצור פערי אינפורמציה גדולים יותר.

הצורך במערכת גבייה היררכית נובע מאסימטריית מידע טבעית בין אליטה ממסה לאליטה ממוסה. התמריץ של הקהילות הוא לשמר את המערכת כדי להבטיח חבות מס נמוכה יותר באופן קולקטיבי ולהשיג שירותים מסוימים מהשלטון. עם זאת, ככל ששחקן קטן יותר, יש לו תמריץ גדול יותר לרמות ולהסתיר את חבותו במס, שכן במקרה של קריסת המערכת הוא יפסיד פחות. בעוד שניתן לאכוף סנקציות נגד יחידים שמרמים, קשה יותר לעשות זאת נגד קבוצה גדולה מחשש למרד והתנגדות. 

הטענה שלנו מנסה לבחון את התיאוריות השונות לגבי חבות המס של הקהילות, בהתאם לממצאים של קליק המתארים "התפוררות" של המערכת ההיררכית. אנו טוענים כי ככל שיש יותר שכבות היררכיות, המס המדווח יהיה נמוך יותר, על אף היחלשות עוצמת ההשפעה ככל שמתרחקים ממרכז הכח בהתאם למודל של אסמוגלו. בנוסף, אנו שולטים על שיעור המס ששולם על ידי הקהילה מכלל חבריה, בהתאם לתיאוריה של אסמוגלו הטוענת שיעילות האכיפה עולה ככל שלאוכפים יש תמריץ לאכוף.

אנו מנצלים את העובדה שהמערכת ההיררכית של הקהילות לא הייתה זהה, חלק מהקהילות הראשיות קיבלו ייצוג אוטונמי כישות כלכלית עצמאית שאינה כפופה ליחידה אחרת, בעוד שקהילות ראשיות אחרות היו כפופות לקהל איזורי ('גליל'),  החלוקה ל'גלילות' מקבילה למערכת פדרלית שבה ערים מרכיבות גוף כמו פרובינציה או state והגוף מייצג אותם במסגרת פרלמנטרית רחבה יותר וייצוג 'אוטונמי' מקביל לנציגות ישירה של העיר או המחוז, ללא תוספת השכבה ההיררכית. 

החלוקה לגלילים התחילה כחלוקה גיאוגראפית-אדמינסטרטיבית שבה נטל המס חולק מטעמי נוחות בין קהילות, אך אך בשל המשמעות שהייתה לחלוקה הזו במונחי שותפות בנטל המס, חלוקה ברווחי חכירה, קבלת צ'ארטרים ופריבלגיות מסחר, קהילות מסויימות בחרו להתאגד ולהרכיב ישות כלכלית עצמאית שאינה כפופה לקהל איזורי ומייוצגת ישירות בוועד המרכזי. "ועד 4 ארצות" הורכב למעשה-22 ישויות, כ-10 קהילות עם ייצוג מחוזי, ו-7 קהילות עם ייצוג אוטונמי, מספר הערים המרכיבות את הקהל היה דומה בכל אחת מהקהילות. קהילות הכפופות ביותר היו עיירות הכפופות לקהל עירוני, הכפוף ל'קהילה ראשית' שמיוצג במסגרת 'גליל'. ביניהם,  המשמעות היא, שחלק מהקהילות הראשיות היו נתונות למערכת עם שכבה היררכית נוספת, וחלקן יוצגו באופן ישיר.

הנחה המרכזית של האמידה האמפירית שלנו היא שמפקד האוכלוסין משקף זעזוע אקסוגני לא-סימטריה באינפורמציה בין השלטון לקהילות, זעזוע שניתן לאמוד באמצעותו את ההפרש בין האוכלוסיה האמיתית לאוכלוסיה המדווחת. אנו לא מניחים שהמפקד מדוייק או שאינו מושפע מהמורכבות ההיררכית, אלא  שההפרש בין תשלום המיסים בשנת 1763 ובין המפקד של 1764 משקף עבור כל קהילה בקואליציה, את מידת היכולת שלה להסתיר את מספר הנפשות שלה בכפוף למערכת ההיררכית. תחת הנחה זו אנו משערים שעצמת הא-סימטריה באינפורמציה גדולה יותר ככל שהריבוד ההיררכי גדול יותר. ניתן לנמק את ההשערה כעלות קבועה של פיקוח חיצוני שעולה עם כל שכבה היררכית נוספת עקב הצורך לפקח על יחידה כלכלית נוספת ועליה בעלות האינפורמציה עקב מורכבות המערכת, לחלופין, ניתן להתמקד בעלות קבועה שמקורה בסוכן, כאשר מקור ההפרש הוא בכך שהעלות מתחלקת על פני מספר קטן יותר של מפוקחים. האסטרטגיה האמפירית שלנו מנצלת את העובדה שהמערכת ההיררכית התבטלה לפני מפקד האוכלוסין, כמעט מיד לאחר תשלום מס הגולגולת באותה שנה. זה מאפשר לנו להפריד בין אסימטריה באינפורמציה שנבעה מהמערכת ההיררכית לבין העלויות ה'טבעיות' של הסוכן.









\subsection{נתונים}
ראשית, אנו מנצלים את נתוני מפקד האוכלסין ונתוני מס הגולגולת כדי לבנות את המדד המרכזי שלנו לאי-סימטריה במידע על ידי השוואת נתוני האוכלוסייה המדווחים עם האוכלוסייה בפועל כפי שתועדה במפקד.
כדי לבנות את המשתנה התלוי שלנו, אנו משתמשים בנתונים לגבי תשלום מס הגולגולת ב-1764, לצורך כך, אנו משתמשים ברישומי מס הגולגולת עבור השנים 1717-1764 שהתגלו על קליק (2009) בארכיון האוצר הפולני. אנו מנצלים את העובדה שהמס עמד על שני זלוטי לנפש כך שיהיה ניתן להשתמש בהם כדי להשוואת האוכלוסיה בפועל לאוכלוסיה המדווחת.
לאחר מכן, אנו משתמשים בנתוני מפקד האוכלוסין היהודי בפולין-ליטא משנת 1764-1765 כבסיס להשוואה לגבי פוטנציאל גביה ללא המערכת ההיררכית. יחד עם נתוני מס הגולגות של אותה שנה אנו יכולים לחשב מדד א-סימטריה עבור כל קהילה ולהעריך את היקף הדיווח החסר של מערכת הקהילות.

 אנו משתמשים במסד הנתונים של "דתות וזרמים דתיים בכתר הפולני במחצית השנייה של המאה ה-18"  (RCP) \cite{szady_geografia_2023} כדי לייצר משתנה עזר של מרחק מבית הכנסת הקרוב ביותר בשנת 1713. מסד הנתונים מורכב מנתונים גיאו-מרחביים, על כל מוסדות דת שהתקיימו במחצית השניה של המאה ה-18 ומספק נתונים על כ-1600 יחידות גיאוגרפיות שבהן התקיים בית כנסת, כאשר מתוכן, 30 בתי כנסת שייכים ל"קהילה ראשית".

לבסוף, אנו משתמשים בנתוני AtlasFontium \citey{historicalatlaspoland} על הכתר הפולני ליטאי המאה ה-18, כדי ליצור משתני בקרה אודות החלוקה האדמינטרטיבית של פולין לפני 1772 (חלוקת פולין) לפי פרובינציה, עיר ועיירה. 
מדד היררכיה
מדד ההירכיה שלנו מנצל את ההבדלים בריבוד ההיררכי בוועד הקהילות, ראשית, אנו יוצרים משתנה פשוט למיקום היררכי המוגדר כסכום של מספר היחידות הממונות על יחידה כלכלית מסויימת עבור כל איזור. שנית, אנו מנצלים אנו מנצלים את העובדה שהקהילות השונות היו כפופות למערכת היררכית שונה כדי ליצור מדד להיררכיה 'שטוחה' לעומת היררכיה מורכבת. 


מדד המיקום ההיררכי H, מסמן עבור כל יחידה גיאוגראפית את מספר האוכפים הממונים עליה, כאשר יחידה, עיר, עיירה או כפר עם K יחידות ממונות תקבל ציון של K+1.
המדד לריבוד היררכי, מציין את מספר האוכפים המקסימלי האפשרי במסגרת ההיררכית וקבוע בין כל היחידות הנמצאות באותה מסגרת היררכית כאוכפים או ככפופים.
למשל, קהל של עיירה הכפוף לקהל עירוני הכפוף למחוז (קהילה ראשית) הכפוף לפרובינציה (גליל) שאמון על גביית מיסים מכפר מסויים, כפוף ל-3 אוכפים וממונה על נאכף 1, לכן, המדד למיקום ההיררכי יהיה $H=4$, והמדד לריבוד ההיררכי יהיה $HS=5$.

לאחר מכן, אנו מסווגים את היחידות לפי הרמה ההיררכית המקסימלית במבנה: כפיפות מחוזית, כפיפות עירונית, וייצוג ישיר, ויוצרים משתנה דמי HS השווה ל-1 אן רמת ההיררכיה המקסימלית $HS<=3$ ושווה ל-0 אם $H<3$. הסיבה לבחירה של $H<3$ כסף לרמת ההיררכיה היא שזהו הסף המבדיל בין קהילה במערכת 'אוטונמית' לקהילה במערכת 'איזורית'. 


\begin{table}[h]
    \centering
    \renewcommand{\arraystretch}{1.4}
    \begin{tabular}{|c c c c|>{\centering\arraybackslash}m{4cm}|>{\centering\arraybackslash}m{4cm}|>{\centering\arraybackslash}m{4cm}|}
   \hline
    \textbf{מיקום היררכי - H} & \multicolumn{3}{c|}{\textbf{ריבוד היררכי - HS}}  \\ \hline
     & 4 & 3 & 2 \\    \hline
    1 & קהילה ראשית & מערכת אוטונומית & קהילה חוץ-טריטוריאלית \\
    
    2 & קהילה איזורית (גליל) & קהילה עירונית & קהילה כפופה \\
    
    3 & קהילה עירונית & קהילה כפופה & \\
    
    4 & קהילה כפופה & & \\
    \hline
    HS & \multicolumn{2}{c}{0} & 1 \\ \hline
\end{tabular}

    \end{table}
        

%%%%%%%%%%%%%%%%%%%%%%%%%%%%%%%%%%%%%%%%%%%
\subsection{אמידה אמפירית}

ראשית אנו מציגים מודל בסיס של השפעת המיקום ההיררכי על הדיווח במפקד האוכלוסין של 1764, כאשר דיווח המיסים של 1763 משמש עבורנו כמשתנה בקרה שמתאר את האוכלוסייה המדווחת. משתנה ההיררכיה מאפשר לנו לאמוד את ההשפעה של ההיררכיה על האוכלוסייה הלא-מדווחת ולכן מציג מדד לפערי אינפורמציה. נתוני מפקד האוכלוסין משמשים כפרוקסי למספר האוכלוסייה האמיתי, אך אנו לא מניחים שדיווחי מפקד האוכלוסין מדויקים אלא שהם משקפים מידע המתקבל תחת אכיפה לאחר תשלום עלויות האינפורמציה. לכן, ניתן לראות את השפעת ההיררכיה כ"עלויות סוכן" במקרה שהיררכיה גבוהה יותר מתואמת עם דיווחים גבוהים יותר, וכ"פרמיית סוכן" במקרה שההיררכיה מתואמת עם דיווחים נמוכים יותר.

המשוואה שלנו היא:

\begin{align} 	Census_i^{1764} = \beta_0 + \beta_1 H_i + \beta_2 Poll^{1764}_i + \mu \end{align}


במשוואה זו, $PollTax^{1764}_i$ מתאר את תשלומי מס הגולגולת של שנת 1764 לפי דיווח קהילתי לשלטון, $Census^{1764}_i$ הוא מפקד האוכלוסין של שנת 1764 ומשמש כמדד לאוכלוסייה האמיתית, ו-$H$ הוא משתנה קטגוריאלי מ-1 עד 4 לפי מספר היחידות הממונות על יחידה גיאוגרפית.

בעיה מרכזית במודל הבסיס היא שהוא למעשה מציג את ההשפעה של מספר האוכפים ולא את ההשפעה של ההיררכיה. אמידה כזו לא מספקת תמונה טובה של השפעת ההיררכיה על פערי האינפורמציה אלא מתארת תוספת אכיפה. בנוסף, המיקום ההיררכי מתואם עם גודל היחידה, כאשר כפרים יהיו בדרך כלל האחרונים במסגרת ההיררכית, כך שהמדד יתפוס את ההבדל הממוצע באי-הדיווח של ערים ועיירות לעומת כפרים, ולא את ההשפעה של ההיררכיה עצמה על דיווח חסר. כדי לתפוס את ההשפעה של המערכת ההיררכית, עלינו לאמוד את ההשפעה של הוספת שכבה היררכית, כלומר, מעבר ממערכת 'מורכבת' לעומת מערכת 'שטוחה' ולאמוד רק את ההשפעה של הוספת השכבה ההיררכית. לשם כך, אנחנו מנצלים את ההבדל בריבוד ההיררכי בתוך הקהילות היהודיות. המשתנה המסביר שלנו הוא האינטראקציה בין המשתנה שמתאר את מספר האוכפים הממונים על היחידה הגיאוגרפית לבין משתני הדמי של מעבר למערכת "אזורית". אנו מסבירים זאת במודל (2).

המשוואה שלנו היא:

$$ \begin{align} H_{1763}*HS_{1763}= \alpha_o + \alpha_1 NobChange +\alpha2 ComChange +\alpha3 NobChange * ComChange\mu \end{align}$$

$$ PollTax^{1764} = \beta_0 + \beta_1 {H}_{1763} + \beta_2 Cencus^{1765}_i + \beta_3 HS_i + \beta_4 \widehat{H_{1763} * HS} + \epsilon $$



במשוואה זו, $\beta_1$ מייצג את ההשפעה של המיקום ההיררכי על דיווח המיסים של הקהילה בשנת 1764, $\beta_2$ מייצג את שיעור מדווחי המיסים בשנת 1763 שלא הושפעו מהמבנה ההיררכי, $\beta_3$ מייצג את ההשפעה של מעבר ממערכת היררכית אחת לאחרת על אי-דיווח, ו-$\beta_4$ הוא המקדם המעניין שמציין את ההשפעה של מעבר מיחידה מרובדת היררכית ליחידה לא מרובדת היררכית. עליה של X במקדם מציינת אי דיווח של $\beta_4$x.

מודל זה עדיף על מודל הבסיס בכך שהוא מאפשר לנו לנצל את ההבדל במבנה ההיררכי של הקהילות כניסוי טבעי, כאשר ההבדל בין צורות השלטון משמש לבניית קבוצות הטיפול והבקרה. ה"טיפול" מוגדר כמעבר של יחידה גיאוגרפית ממבנה היררכי 'אוטונומי' למבנה היררכי 'מורכב'. קבוצת הטיפול היא הקהילות שהיו נתונות למערכת האזורית, וקבוצת הבקרה שלנו הן הקהילות האוטונומיות שהציגו ריבוד היררכי נמוך יותר והיו חשופות לפחות תמריצים להסתרת מידע על ידי גורמים מתווכים.

הנחת הזיהוי במודל זה היא שאין גורמים שהשפיעו על הדיווח במפקד האוכלוסין ומתואמים שיטתית עם הריבוד ההיררכי של המערכת. עם זאת, קשה להצדיק הנחה כזו, מכיוון שלא מדובר בניסוי טבעי טהור, וייתכן שהיו גורמים חיצוניים שהשפיעו על מידת שיתוף הפעולה של הקהילות. ראשית, ייתכן שגורמים חיצוניים בעלי יכולת השפעה על הרמה ההיררכית של הקהילה יכלו ליצור פערי אינפורמציה. אצילים התערבו לא אחת בהחלטות של ועד הארצות והייתה להם יכולת השפעה על רמת הדיווח והציות של הקהילה. כיוון הסיבתיות בין מבנה היררכי לא-סימטריה באינפורמציה עשוי להיות דו-כיווני, ייתכן שקהילות שנטו להסתיר מידע, כמו קהילות עשירות, אימצו היררכיה מורכבת יותר כאמצעי בקרה לזרימת מידע ושליטה.


%%%%%%%%%%%%%%%%%%%%
\subsection{משתנה עזר}

אנו מנצלים שינוי מדיניות שנעשה על ידי וועד הארצות ב-1713, מדיניות זו קבעה כי כל קהילה שאין בתחומה בית כנסת ונמצאת במרחק של עד 2 מייל מקהילה ראשית, שייכת לקהילה זו. המרחק נקבע באופן שרירותי, תחת עקרונות הלכתיים של 'תחום שבת' ולכן אנו יכולים להשתמש במדיניות הזו כמשתנה עזר, שמאפשר לנו לבודד את ההשפעה של גורמים אנדוגניים ולאמוד רק את ההשפעה של הטיפול, ולפרש את האומד כגורם סיבתי.  

אנו משתמשים בנתוני RCP כדי ליצור משתנה דמי המציין האם קיים בית כנסת בקהילה, ומשתנה רציף המציין את המרחק לבית הכנסת הקרוב ביותר. את המודל אנו מעריכים באמצעות שיטת 2SLS, בשלב הראשון אנו מעריכים את השפעת המרחק לבית הכנסת הקרוב ביותר בשנת 1713 על משתנה האינטראקציה בין הרמה ההיררכית והריבוד ההיררכי. בשלב השני, אנו מעריכים את השפעת רמת ההיררכיה (משתנה העזר) על פער המידע. הנחת הזיהוי היא שהמרחק לבית הכנסת בשנת 1731 משפיע על פער המידע רק דרך השפעתו על הריבוד ההיררכי בשנת 1763.
	
הנחת הזיהוי היא ראשית, הרלוונטיות של משתנה העזר, המשתנה שאנחנו מניחים שקיים מתאם חזק בינו לבין משתנה ההיררכיה, הוא משתנה האינטראקציה בין המרחק לבין השינוי במבנה ההיררכי. הסיבה לשימוש דווקא במשתנה האינטראקציה כמשתנה עזר, היא שאנו מניחים שיהיה קשר מובהק סטטיסטית בין המרחק מבית הכנסת ושייכות של היחידה הגיאוראפית לקהילה לבין הרמה ההיררכית, וסביר להניח שיהיה קשה יותר להצדיק את ההנחה הזו תוך שימוש במשנה המרחק בפני עצמו. 
	ההנחה השניה היא הנחת האקסוגניות, כדי שמשתנה העזר יהיה תקף, צריך שיתקיים שהמרחק מבית הכנסת לא מתואם שיטתית עם משתנה הטעות של מודל (2), כלומר, שהמרחק לא משפיעה על פערי האינפורמציה שלא באמצעות המערכת ההיררכית, בעוד שניתן להניח שהמדיניות הזו נקבעה כדי להכניס קהילות מסויימות תחת מבנה היררכי מסויים, המרחק שנבחר  שרירותי, ומושפע רק מכלל הלכתי, ולכן ניתן להשתמש בו כמשתנה עזר אקסוגני.
	
הבעיה עם הנחת האקסוגניות היא שמשתנה העזר שאנחנו טוענים שמקיים את הנחת הרלוונטיות, לא בהכרח מקיים את הנחת האקסוגניות. משתנה האינטראקציה מאפשר לבודד את ההשפעה הסיבתית של היררכיה על א-סימטריה באינפורמציה על ידי ניצול השונות האקסוגנית שנוצרה על ידי מדיניות 1713, אבל זה יתאפשר רק תחת ההנחה שכל הקהילות בתחום צייתו לחוק. אלא שגם אם החוק נאכף בצורה שוויונית, לא היה ניתן למדוד בצורה מדוייקת את המרחק מבית הכנסת הקרוב ביותר, ויש מרחב שיקול דעת לגבי צירוף או אי-צירוף של הקהילות הקטנות לקהילה הראשית. מתקבל על הדעת שקהילות שהיה להן מה להפסיד ממעבר לצורת שלטון אחרת, התעקשו להישאר בה, ויכלו לתמרן את החוק כך שיענה לרצונם. בנוסף, יתכן על אף שקיום בית כנסת ביחידה גיאוראפית הייתה נתונה לשיקול הדעת של השלטון, בהתאם לצ'ראטרים ולפריבלגיות של הערים שהתקבלו לאורך זמן, עדיין ניתן להניח שקהילות ללא בתי כנסת, הפעילו לחצים לקבלת אישור כזה. 

כדי להתמודד עם הבעיה הזו, אנו מציעים משתנה עזר שניתן להניח בסבירות גבוהה שמקיים את שתי ההנחות, במודל (3) אנו מציעים להשתמש במשנה האינטראקציה בין המרחק מבית הכנסת הקרוב ביותר בשנת 1713, לבין רמת ההיררכיה (H) של 1763, באופן הזה, אנחנו יכולים להניח בסבירות גבוהה שהמשתנה רלוונטי, משום שקהילות קטנות יותר היו מרוחקות מבית הכנסת, ולכן המרחק השפיע בעיקר על הקהילות הקטנות, וקיומן של קהילות קטנות מתואם עם ריבוד היררכי מורכב יותר. כלומר, ככל שהמערכת ההיררכית מפוצלת יותר, כך יהיו בה יותר יחידות קטנות, וככל שיהיו בה יותר יחידות קטנות, סביר להניח שהמרחק שלהן מבית הכנסת הקרוב ביותר יהיה גדול יותר. 
בנוסף, קל יותר להניח שהנחת האקסוגניות מתקיימת, משום ש-H עצמו נכלל כמשתנה בקרה ברגרסיה. על ידי הכללת H כמשתנה בקרה, אנו יכולים לבודד את ההשפעה של המרחק מבית הכנסת דרך משתנה האינטראקציה בלבד, ולצמצם את הסיכוי שהאינטראקציה בין המרחק לבין H משפיעה על התוצאות במפקד האוכלוסין שלא דרך הריבוד ההיררכי.
	
	
עדיין קיים חשש שהמרחק עצמו אינו אקסוגני, כך שערים מרוחקות נטו להסתיר מידע וגם 'לרדת מתחת לרדאר' ולהסתיר את עצמן. בעוד שניתן ליצור משתנה דמי השווה לאחד רק עבור קהילות הנמצאות בדיוק במגבלת 2 מייל, משתנה זה ככל הנראה לא יענה על הנחת הרלוונטיות. אנו יוצרים משתנה דמי שמאפשר טווח מרחק מסוים 'חלון' השווה לאחד עבור מרחק הגדול מ-1000 מטר וקטן מ-3000 מטר. את משתנה הדמי אנו מחליפים במשתנה הרציף שמוצג במודל (3.1). הנחת הזיהוי שלנו היא שלא קיים גורם נוסף המשפיע על האינטראקציה בין מרחק מבית הכנסת והרמה ההיררכית ובין פערי האינפורמציה, שלא דרך הריבוד ההיררכי.
המשוואה שלנו היא:
\begin{align}
Hierarchy_{1763} = \alpha_0 + \alpha_1 SynDis + \alpha_2 ComChange + \alpha_3 Dis * ComChange + \mu
\end{align}
 

במשוואה זו, $\alpha_1$ מייצג את ההשפעה של מרחק מבית הכנסת על רמת ההיררכיה, $\alpha_2$ מייצג את ההשפעה הממוצעת של שינוי בעלות קהילתית על רמת ההיררכיה, ו-$\alpha_3$ הוא המקדם המעניין שמייצג קהילות שהמרחק שלהן מבית הכנסת השפיע על רמת ההיררכיה שלהן. עליה במקדם מציינת את המרחק של הקהילה מבית הכנסת.


\subsection{רובסטיות}

במודל האחרון שהצגנו, הטווח שאנחנו מאפשרים נבחר באופן שרירותי כמרחק סביר שמאפשר לתפוס את ההשפעה של המדיניות, תוך הקטנת החשש לאקסוגניות. ככל שנגדיל את הטווח שאנחנו מאפשרים יהיה קשה יותר להצדיק את הנחת האקסוגניות, ככל שנצמצם אותו, יהיה קשה להראות רלוונטיות. כדי לבחון את רגישות התוצאות שלנו אנו בוחנים את הרגישות על סדרה של טווחים, במרחק של 500,1000,1500 מהגבול של 'תחום שבת'. אנו מצפים שההשפעה תלך ותפחת, במונחי השפעה על המקדם או במונחי מובהקות סטטיסטית אבל שהשינויים יהיו קונסיסטנטיים. 

כדי לבחון אם התוצאות אכן משפיעות דרך המערכת ההיררכית הקהילתית, ולא דרך גורמים אדמיניסטרטיביים, אנו מריצים את הרגרסיה של משתנה העזר על קהילות של יהודים קראים, קהילות אלו לא היו כפופות לחוק ההלכתי ולא נכללו במערכת הקהילתית, אבל החוק הפולני כלל אותם במסגרת החלוקה האדמיניסטרטיבית היהודית. במסגרת הצעת המחקר אנו לא יכולים לומר כמה קהילות כאלו התקיימו, ועבור כמה יש נתונים, אולם קיימות קהילות גדולות (למשל, טרוקי) שעבורן יש נתונים, אנו מצפים שעבור קהילות אלו התוצאות יראו משתנה לא מובהק.

כדי לבחון האם התוצאות רגישות לבחינת משתנה העזר שלנו, אנו מציעים משתנה עזר אחר, הבוחן את החלוקות במערכת האדמיניסטרטיבית הפולנית, שהתבצעו עקב ירושה, מלחמות, או רכישות של אצילים, המעבר של יחידה כלכלית מאציל אחד לאציל אחר, השפיעה גם על ההשתייכות של הקהילה בתוך המערכת האדמינטרטיבית של הקהילות. 

אנו אוספים נתונים כאלו באמצעות נתוני היחידות הכלכליות כפי שמופיעות בספרה של קליק, כדי לייצר משתנה השווה לאחד אם יחידה כלכלית פוצלה, ו-0, אחרת, לאחר מכן, אנו מריצים את משתנה העזר על משתנה ההיררכיה שלנו, וכוללים אותו בתוך הרגרסיה ממודל (1), גם משתנה זה עלול להיות אנדוגני מפני שהנתונים מספקים מידע רק על יחידות שפוצלו לאחר שינוי הבעלות, ולא יחידות שעברו לבעלות אצילים אחרים אך לא פוצלו, כדי לשלוט באפשרות הזו, כך שיתכן שאצילים שהיה להם רצון להסתיר מידע נטו יותר לפצל את היחידות תחת בעלותן, וההשפעה אינה של המבנה ההיררכי, אלא של האצילים. כדי לתקן את האפשרות הזו, אנו משתמשים בנתונים של האטלס ההיסטורי של פולין המכיל מידע על בעלות על כפרים וערים עבור כל המאה ה18, ומאפשר לנו לאתר מקרים של שינוי בעלות על יחידות. אנו מאתרים את כל המקרים בהם התבצע שינוי בעלות בין השנים 1717-1763 ומריצים אותו על משתנה האינטראקציה.

\begin{align}
H_{1764}*HS_{1764}= \alpha_0 + \alpha_1 NobChange +\alpha2 ComChange +\alpha3 NobChange * ComChange\mu
\end{align}
	

במשוואה זו, $H_{1764}$ מייצג את משתנה ההיררכיה של הקהילה בשנת 1763, ו-$HS_{1764}$ הוא משתנה ההיררכיה של הקהילה בשנת 1763. $NobChange$ הוא משתנה דמי של שינוי בעלות על יחידה כלכלית, ו-$ComChange$ הוא משתנה דמי של שינוי בעלות קהילתית. $\alpha_0$ מייצג את האפקטים הקבועים של כל רמת היררכיה, $\alpha_1$ מייצג את ההשפעה של שינוי במבנה האצולה על הרמה ההיררכית, $\alpha_2$ מייצג את ההשפעה הממוצעת של שינוי קהילתי על רמת ההיררכיה, ו-$\alpha_3$  שמייצג יחידות כלכליות שחוו שינוי בשייכות קהילתית עקב שינוי בעלות על הקרקע. $\mu$ מייצג את הטעות.
