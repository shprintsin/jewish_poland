במודל האחרון שהצגנו, הטווח שאנחנו מאפשרים נבחר באופן שרירותי כמרחק סביר שמאפשר לתפוס את ההשפעה של המדיניות, תוך הקטנת החשש לאקסוגניות. ככל שנגדיל את הטווח שאנחנו מאפשרים יהיה קשה יותר להצדיק את הנחת האקסוגניות, ככל שנצמצם אותו, יהיה קשה להראות רלוונטיות. כדי לבחון את רגישות התוצאות שלנו אנו בוחנים את הרגישות על סדרה של טווחים, במרחק של 500,1000,1500 מהגבול של 'תחום שבת'. אנו מצפים שההשפעה תלך ותפחת, במונחי השפעה על המקדם או במונחי מובהקות סטטיסטית אבל שהשינויים יהיו קונסיסטנטיים. 

כדי לבחון אם התוצאות אכן משפיעות דרך המערכת ההיררכית הקהילתית, ולא דרך גורמים אדמיניסטרטיביים, אנו מריצים את הרגרסיה של משתנה העזר על קהילות של יהודים קראים, קהילות אלו לא היו כפופות לחוק ההלכתי ולא נכללו במערכת הקהילתית, אבל החוק הפולני כלל אותם במסגרת החלוקה האדמיניסטרטיבית היהודית. במסגרת הצעת המחקר אנו לא יכולים לומר כמה קהילות כאלו התקיימו, ועבור כמה יש נתונים, אולם קיימות קהילות גדולות (למשל, טרוקי) שעבורן יש נתונים, אנו מצפים שעבור קהילות אלו התוצאות יראו משתנה לא מובהק.

כדי לבחון האם התוצאות רגישות לבחינת משתנה העזר שלנו, אנו מציעים משתנה עזר אחר, הבוחן את החלוקות במערכת האדמיניסטרטיבית הפולנית, שהתבצעו עקב ירושה, מלחמות, או רכישות של אצילים, המעבר של יחידה כלכלית מאציל אחד לאציל אחר, השפיעה גם על ההשתייכות של הקהילה בתוך המערכת האדמינטרטיבית של הקהילות. 

אנו אוספים נתונים כאלו באמצעות נתוני היחידות הכלכליות כפי שמופיעות בספרה של קליק, כדי לייצר משתנה השווה לאחד אם יחידה כלכלית פוצלה, ו-0, אחרת, לאחר מכן, אנו מריצים את משתנה העזר על משתנה ההיררכיה שלנו, וכוללים אותו בתוך הרגרסיה ממודל (1), גם משתנה זה עלול להיות אנדוגני מפני שהנתונים מספקים מידע רק על יחידות שפוצלו לאחר שינוי הבעלות, ולא יחידות שעברו לבעלות אצילים אחרים אך לא פוצלו, כדי לשלוט באפשרות הזו, כך שיתכן שאצילים שהיה להם רצון להסתיר מידע נטו יותר לפצל את היחידות תחת בעלותן, וההשפעה אינה של המבנה ההיררכי, אלא של האצילים. כדי לתקן את האפשרות הזו, אנו משתמשים בנתונים של האטלס ההיסטורי של פולין המכיל מידע על בעלות על כפרים וערים עבור כל המאה ה18, ומאפשר לנו לאתר מקרים של שינוי בעלות על יחידות. אנו מאתרים את כל המקרים בהם התבצע שינוי בעלות בין השנים 1717-1763 ומריצים אותו על משתנה האינטראקציה.

\begin{align}
H_{1764}*HS_{1764}= \alpha_0 + \alpha_1 NobChange +\alpha2 ComChange +\alpha3 NobChange * ComChange\mu
\end{align}
	

במשוואה זו, $H_{1764}$ מייצג את משתנה ההיררכיה של הקהילה בשנת 1763, ו-$HS_{1764}$ הוא משתנה ההיררכיה של הקהילה בשנת 1763. $NobChange$ הוא משתנה דמי של שינוי בעלות על יחידה כלכלית, ו-$ComChange$ הוא משתנה דמי של שינוי בעלות קהילתית. $\alpha_0$ מייצג את האפקטים הקבועים של כל רמת היררכיה, $\alpha_1$ מייצג את ההשפעה של שינוי במבנה האצולה על הרמה ההיררכית, $\alpha_2$ מייצג את ההשפעה הממוצעת של שינוי קהילתי על רמת ההיררכיה, ו-$\alpha_3$  שמייצג יחידות כלכליות שחוו שינוי בשייכות קהילתית עקב שינוי בעלות על הקרקע. $\mu$ מייצג את הטעות.
