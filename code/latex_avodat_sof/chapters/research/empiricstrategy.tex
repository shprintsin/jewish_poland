חלק זה בוחן את הקשר בין מבנים היררכיים לא-סימטריה באינפורמציה בקרב קהילות יהודיות בפולין-ליטא במאה ה-18. בדומה למסגרת התיאורטית שהוצגה על ידי \cy{acemoglu_sustaining_2020}  שבו 'אוכפים פנימיים' בוחרים את מידת האכיפה לפי הרווח שלהם מהאכיפה. במסגרת שלנו, כל קהילה היא 'אוכף' מלבד הקהילה האחרונה בשרשרת ההיררכית, וההשערה שלנו היא שקהילות שאופיינו בריבוד היררכי מורכב יותר, יצרו ליצור פערי אינפורמציה גדולים יותר.

הנחה המרכזית של האמידה האמפירית שלנו היא שמפקד האוכלוסין משקף זעזוע אקסוגני לא-סימטריה באינפורמציה בין השלטון לקהילות, זעזוע שניתן לאמוד באמצעותו את ההפרש בין האוכלוסיה האמיתית לאוכלוסיה המדווחת. אנו לא מניחים שהמפקד מדוייק או שאינו מושפע מהמורכבות ההיררכית, אלא  שההפרש בין תשלום המיסים בשנת 1763 ובין המפקד של 1764 משקף עבור כל קהילה בקואליציה, את מידת היכולת שלה להסתיר את מספר הנפשות שלה בכפוף למערכת ההיררכית. תחת הנחה זו אנו משערים שעצמת הא-סימטריה באינפורמציה גדולה יותר ככל שהריבוד ההיררכי גדול יותר. ניתן לנמק את ההשערה כעלות קבועה של פיקוח חיצוני שעולה עם כל שכבה היררכית נוספת עקב הצורך לפקח על יחידה כלכלית נוספת ועליה בעלות האינפורמציה עקב מורכבות המערכת, לחלופין, ניתן להתמקד בעלות קבועה שמקורה בסוכן, כאשר מקור ההפרש הוא בכך שהעלות מתחלקת על פני מספר קטן יותר של מפוקחים.

אנו מנצלים את העובדה שהמערכת ההיררכית של הקהילות לא הייתה זהה, חלק מהקהילות הראשיות קיבלו ייצוג אוטונמי כישות כלכלית עצמאית שאינה כפופה ליחידה אחרת, בעוד שקהילות ראשיות אחרות היו כפופות לקהל איזורי ('גליל'),  החלוקה ל'גלילות' מקבילה למערכת פדרלית שבה ערים מרכיבות גוף כמו פרובינציה או state והגוף מייצג אותם במסגרת פרלמנטרית רחבה יותר וייצוג 'אוטונמי' מקביל לנציגות ישירה של העיר או המחוז, ללא תוספת השכבה ההיררכית. 

החלוקה לגלילים התחילה כחלוקה גיאוגראפית-אדמינסטרטיבית שבה נטל המס חולק מטעמי נוחות בין קהילות, אך אך בשל המשמעות שהייתה לחלוקה הזו במונחי שותפות בנטל המס, חלוקה ברווחי חכירה, קבלת צ'ארטרים ופריבלגיות מסחר, קהילות מסויימות בחרו להתאגד ולהרכיב ישות כלכלית עצמאית שאינה כפופה לקהל איזורי ומייוצגת ישירות בוועד המרכזי. "ועד 4 ארצות" הורכב למעשה-22 ישויות, כ-10 קהילות עם ייצוג מחוזי, ו-7 קהילות עם ייצוג אוטונמי, מספר הערים המרכיבות את הקהל היה דומה בכל אחת מהקהילות. קהילות הכפופות ביותר היו עיירות הכפופות לקהל עירוני, הכפוף ל'קהילה ראשית' שמיוצג במסגרת 'גליל'. ביניהם,  המשמעות היא, שחלק מהקהילות הראשיות היו נתונות למערכת עם שכבה היררכית נוספת, וחלקן יוצגו באופן ישיר.

ההנחה היא שמערכת שבה הריבוד ההיררכי רב יותר, יוצרת אפשרות גדולה יותר לפערי אינפורמציה שמקורן בעלויות סוכנות ובמורכבות הפיקוח. במערכת היררכית, היכולת של סוכנים לנצל את המערכת לטובותם גדולה יותר בהשוואה למערכת 'שטוחה' והשליטה באינפורמציה קלה יותר. לכן, אנו מניחים שקהילות אוטונומיות יציגו פערי אינפורמציה נמוכים יותר בהשוואה לקהילות איזוריות, גם כאשר מספר האוכפים קבוע. 