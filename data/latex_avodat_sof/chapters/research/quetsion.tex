\section{strategy}

המחקר הכלכלי בנוגע לשיתוף פעולה באמצעות קואליציה ואכיפה פנימית ענף למדי, למשל, ההסדרים השיתופיים בין הסוחרים המגריביים וסוכניהם מעבר לים \cite{greif_contract_1993} ונורמות האכיפה הפנימית בין חוואים ובעלי קרקעות \e{\cite{ellickson_order_1991}}. לעומת זאת, המחקר על התלות בין אכיפה חיצונית ואכיפה פנימית הוא בעיקר תיאורטי. הניתוח שלנו מציע מבחן אמפירי לגבי 
היעילות של אכיפה פנימית במסגרת של א-סימטריה באינפורמציה תחת קואליציה עם היררכיה, כאשר, אנו משתמשים בנתונים של המפקד כמדד לא-סימטריה באינפורמציה שנגרמת על ידי הקואליציה.

הטיעון התיאורטי שלנו מבוסס על בעיית "סיכון מוסרי" \e{(Moral Hazard)} במערכת היררכית, שבה היכולת של סוכנים לנצל את המערכת לטובתם גדולה יותר בהשוואה למערכת "שטוחה", והשליטה באינפורמציה קלה יותר. אנו טוענים כי ככל שהמערכת הופכת היררכית יותר, האפקטיביות של האכיפה פוחתת, בעיית ה-\e{Moral Hazard} גדלה, והמידע של המנהלים מצטמצם עקב שכבות נוספות של פיקוח.  
בנוסף, קיימים יחסי מיקוח בין האוכפים הפנימיים וחברי הקואליציה,  בדומה למסגרת התיאורטית שהוצגה על ידי \cy{acemoglu_sustaining_2020}  שבו 'אוכפים פנימיים' בוחרים את מידת האכיפה לפי הרווח שלהם מהאכיפה. במסגרת שלנו, כל קהילה היא 'אוכף' מלבד הקהילה האחרונה בשרשרת ההיררכית, וההשערה שלנו היא שקהילות שאופיינו בריבוד היררכי מורכב יותר, יצרו ליצור פערי אינפורמציה גדולים יותר.

הצורך במערכת גבייה היררכית נובע מאסימטריית מידע טבעית בין אליטה ממסה לאליטה ממוסה. התמריץ של הקהילות הוא לשמר את המערכת כדי להבטיח חבות מס נמוכה יותר באופן קולקטיבי ולהשיג שירותים מסוימים מהשלטון. עם זאת, ככל ששחקן קטן יותר, יש לו תמריץ גדול יותר לרמות ולהסתיר את חבותו במס, שכן במקרה של קריסת המערכת הוא יפסיד פחות. בעוד שניתן לאכוף סנקציות נגד יחידים שמרמים, קשה יותר לעשות זאת נגד קבוצה גדולה מחשש למרד והתנגדות. 

הטענה שלנו מנסה לבחון את התיאוריות השונות לגבי חבות המס של הקהילות, בהתאם לממצאים של קליק המתארים "התפוררות" של המערכת ההיררכית. אנו טוענים כי ככל שיש יותר שכבות היררכיות, המס המדווח יהיה נמוך יותר, על אף היחלשות עוצמת ההשפעה ככל שמתרחקים ממרכז הכח בהתאם למודל של אסמוגלו. בנוסף, אנו שולטים על שיעור המס ששולם על ידי הקהילה מכלל חבריה, בהתאם לתיאוריה של אסמוגלו הטוענת שיעילות האכיפה עולה ככל שלאוכפים יש תמריץ לאכוף.

אנו מנצלים את העובדה שהמערכת ההיררכית של הקהילות לא הייתה זהה, חלק מהקהילות הראשיות קיבלו ייצוג אוטונמי כישות כלכלית עצמאית שאינה כפופה ליחידה אחרת, בעוד שקהילות ראשיות אחרות היו כפופות לקהל איזורי ('גליל'),  החלוקה ל'גלילות' מקבילה למערכת פדרלית שבה ערים מרכיבות גוף כמו פרובינציה או state והגוף מייצג אותם במסגרת פרלמנטרית רחבה יותר וייצוג 'אוטונמי' מקביל לנציגות ישירה של העיר או המחוז, ללא תוספת השכבה ההיררכית. 

החלוקה לגלילים התחילה כחלוקה גיאוגראפית-אדמינסטרטיבית שבה נטל המס חולק מטעמי נוחות בין קהילות, אך אך בשל המשמעות שהייתה לחלוקה הזו במונחי שותפות בנטל המס, חלוקה ברווחי חכירה, קבלת צ'ארטרים ופריבלגיות מסחר, קהילות מסויימות בחרו להתאגד ולהרכיב ישות כלכלית עצמאית שאינה כפופה לקהל איזורי ומייוצגת ישירות בוועד המרכזי. "ועד 4 ארצות" הורכב למעשה-22 ישויות, כ-10 קהילות עם ייצוג מחוזי, ו-7 קהילות עם ייצוג אוטונמי, מספר הערים המרכיבות את הקהל היה דומה בכל אחת מהקהילות. קהילות הכפופות ביותר היו עיירות הכפופות לקהל עירוני, הכפוף ל'קהילה ראשית' שמיוצג במסגרת 'גליל'. ביניהם,  המשמעות היא, שחלק מהקהילות הראשיות היו נתונות למערכת עם שכבה היררכית נוספת, וחלקן יוצגו באופן ישיר.

הנחה המרכזית של האמידה האמפירית שלנו היא שמפקד האוכלוסין משקף זעזוע אקסוגני לא-סימטריה באינפורמציה בין השלטון לקהילות, זעזוע שניתן לאמוד באמצעותו את ההפרש בין האוכלוסיה האמיתית לאוכלוסיה המדווחת. אנו לא מניחים שהמפקד מדוייק או שאינו מושפע מהמורכבות ההיררכית, אלא  שההפרש בין תשלום המיסים בשנת 1763 ובין המפקד של 1764 משקף עבור כל קהילה בקואליציה, את מידת היכולת שלה להסתיר את מספר הנפשות שלה בכפוף למערכת ההיררכית. תחת הנחה זו אנו משערים שעצמת הא-סימטריה באינפורמציה גדולה יותר ככל שהריבוד ההיררכי גדול יותר. ניתן לנמק את ההשערה כעלות קבועה של פיקוח חיצוני שעולה עם כל שכבה היררכית נוספת עקב הצורך לפקח על יחידה כלכלית נוספת ועליה בעלות האינפורמציה עקב מורכבות המערכת, לחלופין, ניתן להתמקד בעלות קבועה שמקורה בסוכן, כאשר מקור ההפרש הוא בכך שהעלות מתחלקת על פני מספר קטן יותר של מפוקחים. האסטרטגיה האמפירית שלנו מנצלת את העובדה שהמערכת ההיררכית התבטלה לפני מפקד האוכלוסין, כמעט מיד לאחר תשלום מס הגולגולת באותה שנה. זה מאפשר לנו להפריד בין אסימטריה באינפורמציה שנבעה מהמערכת ההיררכית לבין העלויות ה'טבעיות' של הסוכן.





