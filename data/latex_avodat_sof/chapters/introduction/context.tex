
אחד התפקידים המרכזיים של ועד הקהילות היה גביית מיסים לטובת השלטון. בשל העדר מערכת אפקטיבית למיסוי הכנסה, השיטה הנפוצה לגביית מיסים וסוחרים הייתה מס גולגולת, מס שטוח שהוטל על כלל האוכלוסיה. היחודיות של ועד הקהילות היא בכך שהוא אפשר גביית מיסים קולקטיבית על-לאומית בין יהודי *בפולין, ליטא, אוקראינה ובלארוס, מאמצע המאה ה-16, הועד החל לשמש כנציג יהודי פולין שאחראי על גביית המס וחלוקתו בין כלל יהודי הכתר. גביית המיסים נעשתה באמצעות חכירת המס - הוועד הציע תשלום מסויים לכתר, הכתר הסכים להחכיר את המס תמורת התשלום, והוועד גבה את המס וחילק אותו כראות עיניו, חלק מהמיסים הועברו לטובת השלטון והשאר למערכת הקהילתית. 

המבנה השלטוני של האיחוד הפולני-ליטאי התבסס על שיטה חצי-פיאודלית, שבראשה עמדה מונרך שנבחר על ידי אצילים בעלי קרקעות ('שלאכטה'), סמכויות המונרך היו מוגבלות עקב הכח המשמעותי של האצולה הגבוהה - המגנאטים.,המגנטים שלטו באחוזות ענק, בגודל של פרובינציה או מחוז, החזיקו צבאות פרטיים והיו בעלי השפעה משמעותית על הכתר ועל האצילים הכפופים להם. האצולה הזוטרה, כללה קבוצה רחבה יותר של אצילים בעלי אחוזות קטנות יותר, שהחזיקו יחד עם המגנאטים בסמכות המחוקקת  באמצעות הסיים (הפרלמנט) תחת עקרון של שוויון האצולה, המגנאטים היו בעלי חזקים משמעותית מהאצולה הנמוכה, בשליטתם בצבא ובכלכלה אך כל אחד מהאצילים החזיק בזכות ווטו, כך שהשפעת המגנאטים באמצעות הסיים הייתה יחסית נמוכה.

הקהילות הורכבו מ'קהילות ראשיות' שהיו חופפות, אך לא זהות, לחלוקה האדמינסטרטיבית של פולין,המחוזות הגדולים בפולין (Wiowodstwo) היו בבעלות של אצולה גבוהה (מגנאטים), לעומת זאת, הערים והכפרים היו נתונים לשליטת אצולה נמוכה. חלוקת הכח בין האצילים הייתה כזו, שהיו מעט אצילים גבוהים והרבה אצילים 
במערכת הפולנית, מחוזות או "וויוודויות" נשלטו לעתים קרובות על ידי המגנאטים, בעוד שעיירות וכפרים היו תחת סמכותה של האצולה הזוטרה. הקהילות היהודיות שיקפו את חלוקת הכוח: ועד הקהילות הורכב מנציגים של "ועדי גלילות" המקבילים לויווידות, ועדי הגלילות הורכבו מ"קהילות ראשיות" שכל אחת מהן שלטה על קהילות עירוניות, הקהילות העירוניות שלטו על "סביבות" שהורכבו מכפרים ומפריפריה עירונית.

בעוד שקיום הוועד המרכזי היה אינטרס של המגנאטים, האינטרס של השלאכטה הנמוכה הייתה שחלק גדול יותר מהמיסים יועבר לטובתם באמצעות החכרת אחוזות וכפרים, האצולה הנמוכה הצליחה לנצל את הסיים לטובתה ולשחוק לאורך זמן את גובה המיסים שהקהילות התחייבו עליהן, כך שהמס נותר קבוע במשך תקופה ארוכה גם לאחר ירידת ערך המטבע.  המבנה הקהילתי חושף למעשה דינמיקה של משא ומתן בין אצילים חזקים שהסתמכו על הוועד כאמצעי גביה, ואצילים חלשים, שהסתמכה על הקהילות היהודיות ליציבות כלכלית ותמיכה מנהלית. \e{\citep{wegrzynek_agreements_2018,teller_tradition_2011}}

מאז 1717, לאחר התחזקות המגנאטים, נקבע כי וועד הקהילות לא משמש כגוף שגובה את המס, אלא רק כגוף שמעריך את גובה נטל המס של כל קהילה. מס הגולגולת היהודי היה חלק ממערך מיסים בשם "היבֶּרְנַה" שנועד לכיסוי הוצאות הצבא. מתוך חשש לניצול לרעה של המערכת ההיררכית כך שקהילות מסויימות ישלמו חלק משמעותי יותר בהכנסות, הסיים קבע כי הערכת מיסי הגולגולת צריכה להיות "הערכה צודקת", כלומר, כזו שמציינת באופן מדוייק, ולכל הפחות פרופוציונאלי, את מספר האנשים בקהילה, נציגי הקהילות היו צריכים להישבע על כך. 

מערכת זו פעלה עד 1764 ואופיינה בחוסר יציבות מתמיד ובתלונות מצד יחידות כלכליות קטנות על מיסוי יתר, \citep{kazmierczyk_permanent_2022}  כינה את התקופה הזו "משבר מתמיד" בשל מצבים חוזרים ונשנים של 'מרד' של קהילות. ב-1764 לאחר התחזקות השלטון המרכזי וביטול זכות הווטו של המגנאטים, החליט השלטון המרכזי לבטל את ועדי הקהילות בפולין ובליטא ולעבור לגביה ישירה מהקהילות ללא תיווך של הוועד. מיד לאחר ההחלטה על ביטול מנגנון הוועד, נערך מפקד אוכלוסין בקרב כלל היהודים בתחומי האיחוד
המפקד חשף פערים משמעותיים בין האוכלוסייה היהודית בפועל לבין הנתונים שדווחו על ידי הקהילות, מה שהצביע על דיווח חסר נרחב והעלמות מס. בתגובה, השלטון פעל לפירוק הקהילות ולהטלת שליטה ישירה על הקהילה היהודית, ובכך סיים את עידן האוטונומיה היהודית בפולין-ליטא.




