קיים קונצנזוס נרחב בנוגע לקשר  בין הקמת "ועד ד ארצות" בפולין ו"ועד המדינה" בליטא לבין גביית מס הגולגולת. חוקרים רבים, כגון מסכימים כי ביטול מוסדות אלה נבע מאי-אפקטיביות של המוסדות בגביית המס. עם זאת, קיימת מחלוקת לגבי הסיבות שהובילו לירידה באפקטיביות של המערכת בגביית המיסים. ניתן לסווג את הגישות השונות בנושא לשתי אסכולות מרכזיות: "השערת הקונפליקט" ו"השערת הזעזוע החיצוני".

החוקרים המחזיקים בהשערת הקונפליקט מתמקדים בגורמים חברתיים ששחקו לאורך זמן את השוויון בנטל המס בין הקהילות, המחלוקת לגבי כיוון ההשפעה ומוקד ההשפעה מחלקת את התומכים בהשערה זו לשני זרמים עיקריים: גישת ה"אליטות המתחרות" וגישת "ביזור הכוח". לפי גישת האליטות המתחרות, שמיוצגת על ידי \cite{teller_early_2010-1} ו\cite{kazmierczyk_permanent_2022}, יחסי הכוח בין האצולה הפולנית לקהילות היהודיות היו במידה רבה יחסים של תלות וניצול, והקונפליקטים הפנימיים בקרב האליטה הובילו לקריסת המערכת. לעומת זאת, גישת ביזור הכוח, שמיוצגת על ידי \cite{kalik_scepter_2009}, רואה ביחסים אלה יחסי מיקוח, כאשר הכוח עבר בהדרגה מאליטה מצומצמת לחוכרים פרטיים ולקהילות קטנות שהיוו קבוצות לחץ.

גישת הזעזוע החיצוני, הכוללת את היילפרין ו-\cite{hundert_jews_2004}, מתמקדת בגורמים חיצוניים בלתי תלויים בחלוקת הנטל כמו מרד הקוזקים של 1648 שיצר שחיקה הדרגתית ביכולת הפיננסית של המערכת, או שינויים תרבותיים אקסוגניים כגון עליית התנועה הקבלית. כמובן שההשערות הללו לא מוכרחות להיות תחליפיות ויכולות להיות משלימות, למשל, @דינור (דינבורג) טען כי ניתן לראות התפרקות הקהילות את ראשיתה של תנועת החסידות שנולדה כביטוי למלחמת מעמדות ול'מרד' עממי שהביא לקריסה. ולמעשה, טלר עצמו, מייחס את התעצמות הקונפליקט לזעזוע חיצוני של עליית מחירי הדגנים שהעצים את ההתנגדות של האצולה הנמוכה למיסוי.

בהתאם לטיעון התיאורטי שיוצג בפרק הבא, השערת המחקר שלנו היא גישת ביזור הכוח במסגרת השערת הקונפליקט, התומכת בעמדתה של קליק. כפי שנציג בהמשך, אנו טוענים כי המורכבות ההיררכית הן של השלטון הפולני והן של השלטון היהודי היא שהובילה לאיבוד היעילות של מערך האכיפה הפנימי, וכיוון ההשפעה של ביזור ההיררכיה על פערי המידע גדל ככל שהמורכבות ההיררכית עלתה. ממצא של קשר סיבתי חיובי מובהק בין היררכיה לפערי מידע יתמוך בגישתה של קליק, בעוד שקשר שלילי יתמוך בגישת טלר. היעדר קשר יתמוך בגישות הזעזוע החיצוני.


